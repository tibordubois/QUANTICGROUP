\section*{Avant-propos}

L’émergence de l’informatique quantique, développée à la fin du siècle dernier, nous mène à de nombreuses avancées algorithmiques et computationnelles. A la différence d’un ordinateur classique, un ordinateur quantique fait recours aux principes indéterministes de la mécanique quantique permettant à la machine de réaliser plusieurs opérations en parallèle. 
Cela donne lieu à la création d'algorithmes quantiques qui nous permet d’aborder des problèmes non-envisageables auparavant avec les algorithmiques classiques. 
L’inférence dans les modèles graphiques, les réseaux bayésiens en particulier, est connue pour être un problème NP-difficile, nécessitant des algorithmes sophistiqués. Ce projet vise à explorer l’application de la programmation quantique à ce domaine spécifique.
\\
Dans un premier temps, nous réaliserons une présentation succincte des réseaux bayésiens. Ensuite, sans préoccupation de la physique quantique à l’issue des machines quantiques, nous introduirons les rudiments de l’informatique quantique. 