\section{Experimentation}

Il s'avère qu'à l'état actuel du développement de la bibliothèque Qiskit en juin 2024, IBM ne propose aucun moyen intégré pour déterminer le temps d'exécution théorique d'un circuit quantique. Or, étant donné que notre projet se concentre sur l'amélioration de la complexité de l'inférence quantique, nous avons dû concevoir un protocole afin d'obtenir une estimation des temps d'exécution de nos circuits.

\subsection{Protocole}

Étant donné qu'un circuit quantique est constitué d'une séquence d'opérations de portes quantiques, une estimation du temps d'exécution se réalise en calculant individuellement le temps requis pour chaque porte, puis en agrégeant ces temps pour obtenir un résultat final.
Notre estimation s'effectue en comptant d'abord le nombre de portes $A$ et de portes $G$ utilisées dans un log, puis en leur attribuant les temps d'exécution respectifs.
\\
Pour obtenir le temps d'exécution de $A$ et $G$, nous avons suivi les étapes suivantes :
\begin{itemize}
    \item 
    Détermination de l'ensemble des portes sur le chemin critique : Nous avons identifié l'ensemble des portes présentes sur le chemin le plus long du circuit après transpilation. En raison du parallélisme des machines quantiques, c'est le temps d'exécution de ce chemin critique qui détermine le temps total d'exécution du circuit.
    \item
    Transpilation du circuit : Le circuit quantique de haut niveau est transpilé en un circuit utilisant uniquement des portes élémentaires. Cette étape est cruciale car elle permet de convertir les portes de haut niveau en séquences de portes élémentaires, adaptées à l'architecture du backend utilisé.
    \item
    Attribution des temps d'exécution : En utilisant les temps d'exécution fournis par le backend pour chaque type de porte élémentaire, nous avons attribué ces temps aux portes identifiées sur le chemin critique.
    \item
    Calcul du temps d'exécution total : En additionnant les temps d'exécution des portes élémentaires présentes sur le chemin critique, nous avons obtenu une estimation du temps total d'exécution pour les portes A et G.
\end{itemize}
Il convient de noter que cette estimation est simplifiée et ne prend pas en compte plusieurs aspects qui pourraient influencer le temps d'exécution réel d'un circuit quantique. 
En effet, nous ne prenons pas en considération le temps de préparation d'un état quantique avant passage par porte quantique et le temps de communication des résultat de la mesure. De plus, nous négligeons les temps additionnels des opérations de correction d'erreur quantique (QEC), les overhead de calibrations, les latence du système de contrôle, et le contraintes spécifiques au dispositif de  connectivité et de conception du circuit. 
Tous ces délais supplémentaires peuvent potentiellement avoir un impact sur le calcul exact du temps d'exécution, cependant, nous avons choisi de les omettre dans le but de simplifier la présentation.

\subsection{Limitations computationnelles}

Lors de nos expérimentations, nous avons rapidement pris conscience des contraintes imposées par l'état actuel de la programmation quantique, en particulier les restrictions liées à la simulation des circuits quantiques sur des machines classiques. En effet, nous utilisons le backend qiskit Aer pour effectuer nos simulations qui s'appuie sur l'algèbre linéaire pour représenter les systèmes quantiques.
\\
Cependant, comme mentionné dans la section \ref{IntroQuant}, les opérations agissant sur de multiples qubits utilisent abondamment le produit tensoriel. Or, sachant qu'un qubit réside dans un espace vectoriel de dimension deux, cette opération entraîne une croissance exponentielle en fonction du nombre de qubits utilisés.
\\
Pour $(|\psi_i\rangle)_{i \in [\![1,n]\!]}$ une famille de $n$ vecteurs d'état de $\mathbb{C}^2$ de dimension $2$, leur produit tensoriel $\bigotimes_{i=1}^{n}|\psi_i\rangle$ se trouve dans l'espace produit $(\mathbb{C}^2)^{\otimes n}$ de dimention $2^n$. Cette complexité est encore amplifiée par  les opérateurs unitaires de l'espace produit, dont la représentation matricielle réside dans $(\mathcal{M}_2(\mathbb{C}))^{\otimes n}$,  un espace de dimension $4^n$.
Ainsi, en pratique, nous nous limiterons à la simulation de réseaux binaires comportant moins de 10 variables afin d'obtenir des résultats dans un délai raisonnable.

\subsection{Résultats}
