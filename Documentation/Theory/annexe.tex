\section{Annexe}
\label{annexe}

Nous expliquerons dans cette section comment construire un circuit quantique pour représenter un réseau bayésien. Le procédé consiste de trois étapes: (\cite{quant_rep_BN})
\begin{itemize}
    \item On associe chaque noeud d'un réseau bayésien à un ou plusieurs qubits. (en fonction du nombre d'états du noeud)
    \item On associe les probabilités marginales/conditionnelles de chaque noeud aux amplitudes de probabilités associées aux états du qubit. 
    \item On accorde les amplitudes de probabilité requises des états quantiques à l'aide de portes de rotation. 
\end{itemize}

% A revoir

\noindent Pour calculer l'angle de rotation \(\theta_{V_i}\) passé en argument de \(\mathrm{R_Y}\) qui accorde les probabilités respectives de \(|0\rangle\) et \(|1\rangle\) à une racine \(V_i\) du graphe, on fait recours à la formule suivante:

\begin{equation}
    \theta_{V_i} = 2 \times arctan(\sqrt{\frac{P(V_i=1)}{P(V_i=0)}})
\end{equation}

\noindent Pour un noeud enfant \(V_i\) ayant un ensemble de parent \(Pa(V_i)\) prenant les valeurs \(\Pi_{V_i}\), l'angle de rotation \(\theta_{V_i, \Pi_{V_i}}\) correspondante se calcul ainsi:

\begin{equation}
    \theta_{V_i, \Pi_{V_i}} = 2 \times arctan(\sqrt{\frac{P(V_i=1|Pa(V_i)=\Pi_{V_i})}{P(V_i=0|Pa(V_i)=\Pi_{V_i})}})
\end{equation}

