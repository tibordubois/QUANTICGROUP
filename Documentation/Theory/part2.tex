\section{Amplitude Amplification}
\label{part2}

Amplitude amplification is a generalization of Grover's algorithm that allows searching for multiple solutions, laying the foundation for many quantum algorithms, and furthermore, providing a quadratic optimization of the complexity of many classical algorithms. \cite{brassard2002quantum}

\subsection{Description}
Revisiting the state space $\mathcal{H}$ endowed with its orthonormal basis $(|x \rangle )_{x \in [\![ 0, N-1 ]\!]}$ introduced in section \ref{part1}, the function $f$ is now the indicator of a subset $\{ \omega_k \}_{k \in [\![ 0, M ]\!]} \subseteq [\![ 0, N-1 ]\!]$ of cardinality $M$.
We then define the projection operator $P=\sum_{k = 0}^m |\omega_k \rangle \langle \omega_k |$ which partitions $\mathcal{H}$ into two orthogonal subspaces:
\begin{align*}
	\mathcal{H}_0 &= \mathrm{Ker}P = \mathrm{Vect}(\{ |x\rangle \in \mathcal{H} \ | \ f(x) = 0\}) \\
    \mathcal{H}_1 &= \ \mathrm{Im}P = \mathrm{Vect}(\{ |x\rangle \in \mathcal{H} \ | \ f(x) = 1\})    
\end{align*}
The principle of the algorithm is therefore to perform a series of rotations on an initial state $| \psi \rangle \in \mathcal{H}$ to place it in $\mathcal{H}_1$, the vector space of solutions.
\\
Given a unitary state $| \psi \rangle \in \mathcal{H} = \mathcal{H}_0 \oplus \mathcal{H}_1$, we have the decomposition:
\begin{align*}
    |\psi \rangle = \mathrm{cos}(\theta)| \psi_0 \rangle + \mathrm{sin}(\theta)| \psi_1 \rangle
\end{align*}
With $\theta = \mathrm{arcsin}(||P |\psi \rangle||_2) \in [0, \pi / 2]$ and 
\begin{align*}
|\psi_0 \rangle =&\ \frac{(Id-P) |\psi \rangle}{||(Id-P) |\psi \rangle||_2} \in \mathcal{H}_0 \\ 
|\psi_1 \rangle =&\ \quad \ \, \frac{P |\psi \rangle}{||P |\psi \rangle||_2} \quad \ \ \in \mathcal{H}_1
\end{align*}
the normalized projections of $|\psi \rangle$.
\\
This decomposition gives rise to the subspace $\mathcal{H}_{\psi} = \mathrm{Vect}(|\psi_0 \rangle, \psi_1 \rangle)$. 
\\
We can note that the probability of obtaining the initial vector $|\psi \rangle = \sum_{k=0}^{N-1} |k\rangle$ in $\mathcal{H}_1$ without having executed the algorithm is $\mathrm{sin}^2(\theta) = ||P |\psi \rangle||_2^2 = M/N$, which corresponds to the uniform probability of obtaining a solution by choosing an element at random in $[\![ 0, N-1 ]\!]$. After execution, this probability approaches 1.

\subsection{Algorithm}

We find the same algorithm as in the previous section, this time with the operators:

\begin{align*}
	U_{P} = & \ Id - 2P \\
    U_{\psi} = & \ 2 | \psi \rangle \langle \psi | - Id
\end{align*}

\noindent Setting $Q = U_{\psi} U_{P} $, we have:
\begin{align*}
	Q | \psi_0 \rangle =& \quad \ \, U_{\psi} | \psi_0 \rangle = \ (2 \mathrm{cos}^2(\theta)-1) | \psi_0 \rangle + \ 2 \mathrm{sin}(\theta)\mathrm{cos}(\theta) | \psi_1 \rangle \\
    Q | \psi_1 \rangle =&\ -U_{\psi} | \psi_1 \rangle = -2 \mathrm{sin}(\theta)\mathrm{cos}(\theta) | \psi_0 \rangle + (1 + 2 \mathrm{sin}^2(\theta)) | \psi_1 \rangle
\end{align*}
knowing that $\langle \psi | \psi_0 \rangle = \mathrm{cos}(\theta)$ and $\langle \psi | \psi_1 \rangle = \mathrm{sin}(\theta)$.
\\
Thus in $\mathcal{H}_{\psi}$, the operator $Q$ corresponds to a rotation by angle $2\theta$:
\[
Q =
\begin{pmatrix}
\mathrm{cos}(2\theta) & -\mathrm{sin}(2\theta) \\
\mathrm{sin}(2\theta) & \mathrm{cos}(2\theta)
\end{pmatrix}
\]
Applying the operator $Q$ on the state $| \psi \rangle$ a number $r$ of times, we obtain:
\[ Q^r | \psi \rangle = \mathrm{cos}((2r+1)\theta)| \psi_0 \rangle + \mathrm{sin}((2r+1)\theta)| \psi_1 \rangle\]
The probability of obtaining the desired state upon measurement after $r$ iterations is thus $\mathrm{sin}^2 \left( \left( 2r + 1 \right) \theta \right)$ which is maximized for $r \approx \frac{\pi}{4\theta}$. 
\\
For $\mathrm{sin}(\theta) \ll 1$, we can further approximate the value of $r$ by: \[\frac{\pi}{4\mathrm{sin}(\theta)} = \frac{\pi}{4} \sqrt{\frac{N}{M}} = \mathcal{O}(\sqrt{N})\]
