\clearpage
\section{Quantum Rejection Sampling}

Based on the construction in section \ref{part4}, we leverage amplitude amplification for performing inference using rejection sampling. For a Bayesian network with $k$ variables each having fewer than $m$ parents, classical sampling of the joint distribution operates in $\mathcal{O}(km)$. Consequently, for $P_e$ the probability of observation, the rejection sampling method has an expected time complexity of $\mathcal{O}(km/P_e)$. As the number of observation variables increases, the computational efficiency of the classical algorithm deteriorates especially when $P_e$ becomes small.
\\[5pt]
Rejection sampling operates as an unstructured search for samples within a distribution. Amplitude amplification enhances this process by providing a square root improvement in the efficiency of the search. \cite{low2014quantum}

\subsection{Description}

To simplify the notations, consider a Bayesian network with of $k$ binary states, where $\mathcal{E}$ denotes the set of observed states and $\mathcal{Q}$ the set of target states.
\\[5pt]
Let $(e_i)_{i \in [\![1,|\mathcal{E}|]\!]}$ be the qubits representing the elements of $\mathcal{E}$ and $e = e_1e_2\cdots e_{|\mathcal{E}|}$ the corresponding bit string. Similarly, let $q$ represent the bit string corresponding to $\mathcal{Q}$.
\\[5pt]
The objective of the algorithm is to sample from the distribution $\mathbb{P}(\mathcal{Q}|\mathcal{E}=e)$.
\\[5pt]
Let $A$ be the unitary operator that prepares the q-sample $| \psi \rangle = A |0\rangle ^{\otimes k}$. By rearranging the states of $\mathcal{E}$ appropriately, we can decompose the q-sample into states containing both correct and incorrect observations:
\begin{align*}
	|\psi \rangle =&\ \sqrt{1-P_e}\ |\psi_0\rangle \quad + \sqrt{P_e}\ |\psi_1\rangle \\
	=&\ \sqrt{1-P_e}\ |q\rangle |e_0\rangle + \sqrt{P_e}\ |q\rangle |e_1\rangle
\end{align*}
With $P_e = \mathbb{P}(\mathcal{E}=e)$, $|\psi_0 \rangle \in \mathcal{Q} \otimes \mathcal{E}_0$ and $|\psi_1 \rangle \in \mathcal{Q} \otimes \mathcal{E}_1$.
\\[5pt]
Here, we can see the function $f$ as:
\begin{align*}
    f : \{0,1\}^{|\mathcal{Q}|}\times\{0,1\}^{|\mathcal{E}|} &\longrightarrow \{0,1\} \\
    (q_1,\hdots,q_{|\mathcal{Q}|}, e_1,\hdots,e_{|\mathcal{E}|}) &\longmapsto
 \begin{cases}
 1 \ \mathrm{if} \ e_1 \cdots \, e_{|\mathcal{E}|} = e \\
 0 \ \mathrm{otherwise}
 \end{cases}
\end{align*}
As in the previous sections, $f$ partitions $\mathcal{H} = \mathcal{Q} \otimes \mathcal{E}$ into two orthogonal vector spaces:
\[\mathcal{H} = \mathcal{Q} \otimes (\mathcal{E}_0 \oplus \mathcal{E}_1) = (\mathcal{Q} \otimes \mathcal{E}_0) \oplus (\mathcal{Q} \otimes \mathcal{E}_1)\]
Where $\mathcal{Q} \otimes \mathcal{E}_0$ contains the q-samples of the distribution $\mathbb{P}(\mathcal{Q,E}|\mathcal{E}\neq e)$, and $\mathcal{Q} \otimes \mathcal{E}_1$ of $\mathbb{P}(\mathcal{Q,E}|\mathcal{E} = e)$.
By performing amplitude amplification on the system, we obtain a q-sample of the distribution $\mathbb{P}(\mathcal{Q}|\mathcal{E} = e)$ with high probability.

\subsection{Algorithm}
\label{alg:QuantRS}
\begin{algorithm}[H]
Let $i$ be the iterator of the outer loop:
\begin{itemize}
    \item[1] Initialize $|\psi \rangle = A|0\rangle^{\otimes k}$.
    \item[2] Apply Grover's iteration $2^i$ times: For $j \in [\![0, 2^i-1]\!]$,
    \begin{itemize}
  	 
   	 \item[2.1] Apply to $|\psi_{(j)}\rangle$ the operator $
   	 U_{e} = Id_{\mathcal{Q}} \otimes (Id_{\mathcal{E}} -  2|e \rangle \langle e| )
   	 $
   	 \item[2.2] Apply to $|\psi_{(j)}\rangle$ the operator $U_{\psi} = 2|\psi\rangle \langle \psi | - Id_{\mathcal{H}} = AU_{0}A^{-1}
   	 $
     \\ where $U_{0} = 2|0\rangle \langle 0|^{\otimes k} - Id_{\mathcal{H}}$.
   	 
    \end{itemize}
    \item[] Thus, we end up with $|\psi_{(i+1)}\rangle = AU_{0}A^{-1}U_{e} |\psi_{(i)} \rangle$.
    \item[3] Measure the observation qubits $\mathcal{E}$ of the resulting state $|\psi_{(2^i)} \rangle$: Let $t$ be the result.
	\item[4]
	\begin{itemize}
   	 \item[4.1] If $t=e$: End of the algorithm.
   	 \item[4.2] Otherwise: Return to step 1 and increment $i$ by $1$.
    \end{itemize}
\end{itemize}
\end{algorithm}

\begin{figure}[H]
\centering
\scalebox{1}{
\Qcircuit @C=.6em @R=.6em { \\
	 	\lstick{{\ket{0}} :  } & \multigate{2}{\mathrm{A}} & \multigate{2}{\mathrm{G^{2^{i+1}}}} & \meter & \qw & \qw & \qw \\
        \lstick{\raisebox{0.5em}{\vdots}\hspace{0.7em}} & \nghost{\mathrm{A}} & \nghost{\mathrm{G^{2^{i+1}}}} & & \raisebox{0.5em}{\rotatebox{35}{\vdots}} & & \\
        \lstick{{\ket{0}} :  } & \ghost{\mathrm{A}} & \ghost{\mathrm{G^{2^{i+1}}}} & \qw & \qw & \meter & \qw\\
	 	\lstick{\mathrm{{meas} :  }} & \cw & \lstick{/_{_{n}}} \cw & \dstick{_{_{\hspace{0.0em}1}}} \cw \ar @{<=} [-3,0] & \dstick{_{_{\hspace{0.0em}\hdots}}} \cw & \dstick{_{_{\hspace{0.0em}n}}} \cw \ar @{<=} [-1,0] & \cw\\
\\ }}
\caption{Quantum circuit of algorithm \ref{alg:QuantRS}}
    \label{fig:GroverCirc}
\end{figure}

\begin{figure}[H]
    \centering
\scalebox{.8}{
\raisebox{-1em}{
\Qcircuit @C=.6em @R=.6em { \\
	 	\qw & \multigate{2}{\mathrm{G}} & \qw \\
        \raisebox{0.5em}{\vdots} & \nghost{\mathrm{G}} & \raisebox{0.5em}{\vdots} \\
        \qw & \ghost{\mathrm{G}} & \qw \\
\\ }
\hspace{0.5em}
\raisebox{-2.9em}{\Large =}
\hspace{0.5em}
\Qcircuit @C=.6em @R=.6em { \\
	 	\qw & \multigate{2}{\mathrm{U_e}} & \multigate{2}{\mathrm{A^{\dag}}} & \multigate{2}{\mathrm{U_0}} & \multigate{2}{\mathrm{A}} & \qw \\
        \raisebox{0.5em}{\vdots} & \nghost{\mathrm{U_e}} & \nghost{\mathrm{A^{\dag}}} & \nghost{\mathrm{U_0}} & \nghost{\mathrm{A}} & \raisebox{0.5em}{\vdots} \\
        \qw & \ghost{\mathrm{U_e}} & \ghost{\mathrm{A^{\dag}}} & \ghost{\mathrm{U_0}} & \ghost{\mathrm{A}} & \qw \\
\\ }
}

\hspace{1.5cm}

\raisebox{-0.8em}{
\Qcircuit @C=.6em @R=.8em { \\
	 	\qw & \multigate{2}{\mathrm{U_e}} & \qw \\
        \raisebox{0.5em}{\vdots} & \nghost{\mathrm{U_e}} & \raisebox{0.5em}{\vdots} \\
        \qw & \ghost{\mathrm{U_e}} & \qw \\
\\ }
}

\hspace{0.5em}
\raisebox{-4.1em}{\Large =}
\hspace{1em}

\Qcircuit @C=.6em @R=1em {
\lstick{} & \qw & \qw & \qw & \qw \\
\lstick{} & \raisebox{0.5em}{\vdots} & & \raisebox{0.5em}{\vdots} & \\
\lstick{} & \qw & \qw & \qw & \qw 
\inputgroupv{1}{3}{.8em}{1.4em}{\mathcal{Q}}\\
\lstick{} & \gate{\mathrm{X_{1}^{1-e_1}}} & \ctrl{2} & \gate{\mathrm{X_{1}^{1-e_1}}} & \qw \\
\lstick{} & \raisebox{0.5em}{\vdots} & & \raisebox{0.5em}{\vdots} & \\
\lstick{} & \gate{\mathrm{X_{\scalebox{.7}{$|\mathcal{E}|$}}^{1-e_{\scalebox{.5}{$|\mathcal{E}|$}}}}} & \control\qw & \gate{\mathrm{X_{\scalebox{.7}{$|\mathcal{E}|$}}^{1-e_{\scalebox{.5}{$|\mathcal{E}|$}}}}} & \qw \\
\inputgroupv{4}{6}{0.8em}{2.4em}{\mathcal{E}}\\
}
}
\caption{Circuits of quantum gates used in \ref{fig:GroverCirc}}
    \label{fig:GatesCirc}
\end{figure}

\subsection{Complexity}

Based on the preparation time of a q-sample detailed in the previous section, we derive an expected total complexity of $\mathcal{O}(k2^m/\sqrt{P_e})$ to generate a sample from the distribution $\mathbb{P}(\mathcal{Q}|\mathcal{E}=e)$.
