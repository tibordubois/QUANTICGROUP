\section{Quantum Sampling by Rejection Method}

Based on the construction in section \ref{part4}, we can use amplitude amplification to perform inference using the rejection method. For a Bayesian network with $k$ variables each having fewer than $m$ parents, generating classical samples of the joint distribution is done in $\mathcal{O}(km)$. Thus, for $P_e$ the probability of observation, the rejection method has an expected complexity of $\mathcal{O}(km/P_e)$. The computational performance of the classical algorithm degrades as soon as $P_e$ becomes exponentially small when the number of observation variables increases. In fact, rejection sampling can be seen as an unstructured search for samples in the distribution, and amplitude amplification allows for a square root improvement of this search. \cite{low2014quantum}

\subsection{Description}

To simplify the notations, let us assume that the Bayesian network consists of $k$ binary states, with $\mathcal{E}$ the set of observation states and $\mathcal{Q}$ the target states.
Let $(e_i)_{i \in [\![1,|\mathcal{E}|]\!]}$ be the qubits representing the elements of $\mathcal{E}$ and $e = e_1e_2\cdots e_{|\mathcal{E}|}$ the corresponding bit string. Similarly, let $q$ be the string representing $\mathcal{Q}$.
The goal of the algorithm is thus to sample from the distribution $\mathbb{P}(\mathcal{Q}|\mathcal{E}=e)$.
\\
Let $A$ be the unitary operator that prepares the q-sample $| \psi \rangle = A |0\rangle ^{\otimes k}$. By permuting the states of $\mathcal{E}$ to the right, we have a decomposition of the q-sample with states containing correct and incorrect observations:
\begin{align*}
	|\psi \rangle =&\ \sqrt{1-P_e}\ |\psi_0\rangle \quad + \sqrt{P_e}\ |\psi_1\rangle \\
	=&\ \sqrt{1-P_e}\ |q\rangle |e_0\rangle + \sqrt{P_e}\ |q\rangle |e_1\rangle
\end{align*}
With $P_e = \mathbb{P}(\mathcal{E}=e)$, $|\psi_0 \rangle \in \mathcal{Q} \otimes \mathcal{E}_0$ and $|\psi_1 \rangle \in \mathcal{Q} \otimes \mathcal{E}_1$.
\\
Here, we can see the function $f$ as:
\begin{align*}
    f : \{0,1\}^{|\mathcal{Q}|}\times\{0,1\}^{|\mathcal{E}|} &\longrightarrow \{0,1\} \\
    (q_1,\hdots,q_{|\mathcal{Q}|}, e_1,\hdots,e_{|\mathcal{E}|}) &\longmapsto
 \begin{cases}
 1 \ \mathrm{if} \ e_1 \cdots \, e_{|\mathcal{E}|} = e \\
 0 \ \mathrm{otherwise}
 \end{cases}
\end{align*}
As in the previous sections, $f$ partitions $\mathcal{H} = \mathcal{Q} \otimes \mathcal{E}$ into two orthogonal vector spaces:
\[\mathcal{H} = \mathcal{Q} \otimes (\mathcal{E}_0 \oplus \mathcal{E}_1) = (\mathcal{Q} \otimes \mathcal{E}_0) \oplus (\mathcal{Q} \otimes \mathcal{E}_1)\]
Where $\mathcal{Q} \otimes \mathcal{E}_0$ contains the q-samples of the distribution $\mathbb{P}(\mathcal{Q,E}|\mathcal{E}\neq e)$, and $\mathcal{Q} \otimes \mathcal{E}_1$ of $\mathbb{P}(\mathcal{Q,E}|\mathcal{E} = e)$.
By performing amplitude amplification on the system, we obtain a q-sample of the distribution $\mathbb{P}(\mathcal{Q}|\mathcal{E} = e)$ with high probability.

\subsection{Algorithm}

Let $i$ be the iterator of the outer loop:
\begin{itemize}
    \item[1] Initialize $|\psi \rangle = A|0\rangle^{\otimes k}$.
    \item[2] Apply Grover's iteration $2^i$ times:
    \item[] For $j \in [\![0, 2^i-1]\!]$,
    \begin{itemize}
  	 
   	 \item[2.1] Apply to $|\psi_{(j)}\rangle$ the operator $U_{e}$, also given by:
   	 \begin{align*}
   	 U_{e} = Id_{\mathcal{Q}} \otimes (Id_{\mathcal{E}} -  2|e \rangle \langle e| )
   	 \end{align*}
   	 \item[2.2] Apply to $|\psi_{(j)}\rangle$ the operator $U_{\psi}$:
    	\begin{align*}
   	 U_{\psi} = 2|\psi\rangle \langle \psi | - Id_{\mathcal{H}}
   	 \end{align*}
    	This is also given by:
    	\begin{align*}
    	U_{\psi} = AU_{0}A^{-1}
   	 \end{align*}
    	where $U_{0} = 2|0\rangle \langle 0|^{\otimes k} - Id_{\mathcal{H}}$.
   	 
    \end{itemize}
    \item[] Thus, we end up with $|\psi_{(i+1)}\rangle = AU_{0}A^{-1}U_{\omega} |\psi_{(i)} \rangle$.
    \item[3] Measure the observation qubits $\mathcal{E}$ of the resulting state $|\psi_{(2^i)} \rangle$: Let $t$ be the result.
	\item[4]
	\begin{itemize}
   	 \item[4.1] If $t=e$: End of the algorithm.
   	 \item[4.2] Otherwise: Return to step 1 and increment $i$ by $1$.
    \end{itemize}
\end{itemize}

\subsection{Complexity}

Based on the preparation time of a q-sample detailed in the previous section, we obtain a total expected complexity of $\mathcal{O}(k2^m/\sqrt{P_e})$ to generate a sample from the distribution $\mathbb{P}(\mathcal{Q}|\mathcal{E}=e)$.
