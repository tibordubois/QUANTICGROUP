\section{Échantillonage quantique par la méthode de rejet}

En représentant un réseau bayésien par un circuit quantique, nous pouvons utiliser l'amplification d'amplitude afin de faire de l'inférence à l'aide de la méthode de rejet. \cite{low2014quantum}

\subsection{Description}

Afin de simplifier les notations, supposons que le réseau bayésien est composé de $n$ états binaires, avec $\mathcal{E}$ l'ensemble des états d'observation et $\mathcal{Q}$ les états cibles.
Soient $(e_i)_{i \in [\![1,|\mathcal{E}|]\!]}$ les qubits représentant les éléments de $\mathcal{E}$ et $e = e_{|\mathcal{E}|}...e_2e_1$ la chaîne de bits correspondante. De la même manière définissons $q$ la chaîne représentant $\mathcal{Q}$.
Le but de l'algorithme est ainsi d'échantillonner à partir de la distribution $\mathbb{P}(\mathcal{Q}|\mathcal{E}=e)$.
\\
Soit $A$ l'opérateur unitaire qui prépare le vecteur d'état représentatif du réseau bayésien $| \psi \rangle = A |0\rangle ^{\otimes n}$, appelé \textit{q-échantillon}. En permutant les états de $\mathcal{E}$ à droite, on a une décomposition du q-échantillons avec des états contenant des observations correctes et incorrectes:
\begin{align*}
	|\psi \rangle =&\ \sqrt{1-P_e}\ |\psi_0\rangle \quad + \sqrt{P_e}\ |\psi_1\rangle \\
	=&\ \sqrt{1-P_e}\ |q\rangle |e_0\rangle + \sqrt{P_e}\ |q\rangle |e_1\rangle
\end{align*}
Avec $P_e = \mathbb{P}(\mathcal{E}=e)$, $|\psi_0 \rangle \in = \mathcal{Q} \otimes \mathcal{E}_0$ et $|\psi_1 \rangle \in = \mathcal{Q} \otimes \mathcal{E}_1$.
\\
Ici, on peut voir la fonction $f$ comme:
\begin{align*}
    f : \{0,1\}^{|\mathcal{Q}|}\times\{0,1\}^{|\mathcal{E}|} &\longrightarrow \{0,1\} \\
    (q_{|\mathcal{Q}|},...,q_1, e_{|\mathcal{E}|},...,e_1) &\longmapsto
 \begin{cases}
 1 \ \mathrm{si} \ e_{|\mathcal{E}|},...,e_1 = e \\
 0 \ \mathrm{sinon} \
 \end{cases}
\end{align*}
Comme dans les sections précédentes, $f$ partitionne $\mathcal{H} = \mathcal{Q} \otimes \mathcal{E}$ en deux espaces vectoriels orthogonaux:
\[\mathcal{H} = \mathcal{Q} \otimes (\mathcal{E}_0 \oplus \mathcal{E}_1) = (\mathcal{Q} \otimes \mathcal{E}_0) \oplus (\mathcal{Q} \otimes \mathcal{E}_1)\]
Où $\mathcal{Q} \otimes \mathcal{E}_0$ contient les q-échantillons de la distribution $\mathbb{P}(\mathcal{Q,E}|\mathcal{E}\neq e)$, et $\mathcal{Q} \otimes \mathcal{E}_1$ de $\mathbb{P}(\mathcal{Q,E}|\mathcal{E} = e)$.
En effectuant une amplification d'amplitude sur le système, nous obtenons un q-échantillon de la distribution $\mathbb{P}(\mathcal{Q}|\mathcal{E} = e)$ avec haute probabilité en $O(1/\sqrt{P_e})$.

\subsection{Algorithme}

Soit $i$ l'itérateur de la boucle externe:
\begin{itemize}
    \item[1] Initialiser $|\psi \rangle = A|0\rangle^{\otimes n}$.
    \item[2] Appliquer l'itération de Grover $2^i$ fois:
    \item[] Pour $j \in [\![0, 2^i-1]\!]$,
    \begin{itemize}
  	 
   	 \item[2.1] Appliquer à $|\psi_{(j)}\rangle$ l'opérateur $U_{e}$, donné aussi par:
   	 \begin{align*}
   	 U_{e} = Id_{\mathcal{Q}} \otimes (Id_{\mathcal{E}} -  2|e \rangle \langle e| )
   	 \end{align*}
   	 \item[2.2] Appliquer à $|\psi_{(j)}\rangle$ l'opérateur $U_{\psi}$ :
    	\begin{align*}
   	 U_{\psi} = 2|\psi\rangle \langle \psi | - Id_{\mathcal{H}}
   	 \end{align*}
    	Celui-ci est aussi donnée par:
    	\begin{align*}
    	U_{\psi} = AU_{0}A^{-1}
   	 \end{align*}
    	ou $U_{0} = 2|0\rangle \langle 0|^{\otimes n} - Id_{\mathcal{H}}$.
   	 
    \end{itemize}
    \item[] On se retrouve donc avec $|\psi_{(i+1)}\rangle = AU_{0}A^{-1}U_{\omega} |\psi_{(i)} \rangle$.
    \item[3] Mesurer les qubits d'observation $\mathcal{E}$ de l'état résultant $|\psi_{(2^i)} \rangle$.
	\item[4]
	\begin{itemize}
   	 \item[4.1] Si $\mathcal{E}=e$: Fin de l'algorithme.
   	 \item[4.2] Sinon: Retourner à l'étape 1 en incrémentant $i$ de $1$.
    \end{itemize}
\end{itemize}
