\section{Amplification d'amplitude}
\label{part2}

L'amplification d'amplitude est une généralisation de l'algorithme de Grover qui permet à la recherche de plusieurs solutions, elle pose les bases pour de nombreux algorithmes quantiques, et permet en outre une optimisation quadratique de la complexité de nombreux algorithmes classiques. \cite{brassard2002quantum}

\subsection{Description}
En reprenant l'espace d'états $\mathcal{H}$ muni de sa base orthonormale $(|x \rangle )_{x \in [\![ 0, N-1 ]\!]}$ introduite dans la partie \ref{part1}, la fonction $f$ est à présent  l'indicatrice d'un sous ensemble $\{ \omega_k \}_{k \in [\![ 0, M ]\!]} \subseteq [\![ 0, N-1 ]\!]$ de cardinal $M$.
On définit par la suite l'opérateur de projection $P=\sum_{k = 0}^m |\omega_k \rangle \langle \omega_k |$ qui partitionne $\mathcal{H}$ en deux sous espaces vectoriels orthogonaux:
\begin{align*}
	\mathcal{H}_0 &= \mathrm{Ker}P = \mathrm{Vect}(\{ |x\rangle \in \mathcal{H} \ | \ f(x) = 0\}) \\
    \mathcal{H}_1 &= \ \mathrm{Im}P = \mathrm{Vect}(\{ |x\rangle \in \mathcal{H} \ | \ f(x) = 1\})    
\end{align*}
Le principe de l'algorithme est donc de réaliser une série de rotations sur un état initial $| \psi \rangle \in \mathcal{H}$ afin de le situer dans $\mathcal{H}_1$, l'espace vectoriel des solutions.
\\
Étant donné un état unitaire $| \psi \rangle \in \mathcal{H} = \mathcal{H}_0 \oplus \mathcal{H}_1$, nous avons la décomposition:
\begin{align*}
    |\psi \rangle = \mathrm{cos}(\theta)| \psi_0 \rangle + \mathrm{sin}(\theta)| \psi_1 \rangle
\end{align*}
Avec $\theta = \mathrm{arcsin}(||P |\psi \rangle||_2) \in [0, \pi / 2]$ et 
\begin{align*}
|\psi_0 \rangle =&\ \frac{(Id-P) |\psi \rangle}{||(Id-P) |\psi \rangle||_2} \in \mathcal{H}_0 \\ 
|\psi_1 \rangle =&\ \quad \ \, \frac{P |\psi \rangle}{||P |\psi \rangle||_2} \quad \ \ \in \mathcal{H}_1
\end{align*}
les projections normalisées de $|\psi \rangle$.
\\
Cette décomposition donne lieu au sous espace vectoriel $\mathcal{H}_{\psi} = \mathrm{Vect}(|\psi_0 \rangle, \psi_1 \rangle)$. 
\\
On peut remarquer que la probabilité d'obtenir le vecteur initial $|\psi \rangle = \sum_{k=0}^{N-1} |k\rangle$ dans $\mathcal{H}_1$ sans avoir exécuté l'algorithme est de $\mathrm{sin}^2(\theta) = ||P |\psi \rangle||_2^2 = M/N$, ce qui correspond bien à la probabilité uniforme d'obtenir une solution en choisisant un élément au hasard dans $[\![ 0, N-1 ]\!]$. Après exécution, cette probabilité se rapproche de 1. 

\subsection{Algorithme}

Nous retrouvons le même algorithme de la section précédente, avec cette fois-ci les opérateurs:

\begin{align*}
	U_{P} = & \ Id - 2P \\
    U_{\psi} = & \ 2 | \psi \rangle \langle \psi | - Id
\end{align*}

\noindent En posant $Q = U_{\psi} U_{P} $ on a:
\begin{align*}
	Q | \psi_0 \rangle =& \quad \ \, U_{\psi} | \psi_0 \rangle = \ (2 \mathrm{cos}^2(\theta)-1) | \psi_0 \rangle + \ 2 \mathrm{sin}(\theta)\mathrm{cos}(\theta) | \psi_1 \rangle \\
    Q | \psi_1 \rangle =&\ -U_{\psi} | \psi_1 \rangle = -2 \mathrm{sin}(\theta)\mathrm{cos}(\theta) | \psi_0 \rangle + (1 + 2 \mathrm{sin}^2(\theta)) | \psi_1 \rangle
\end{align*}
sachant que $\langle \psi | \psi_0 \rangle = \mathrm{cos}(\theta)$ et $\langle \psi | \psi_1 \rangle = \mathrm{sin}(\theta)$.
\\
Ainsi dans $\mathcal{H}_{\psi}$, l'opérateur $Q$ correspond à une rotation d'angle $2\theta$:
\[
Q =
\begin{pmatrix}
\mathrm{cos}(2\theta) & -\mathrm{sin}(2\theta) \\
\mathrm{sin}(2\theta) & \mathrm{cos}(2\theta)
\end{pmatrix}
\]
En appliquant l'opérateur $Q$ sur l'état $| \psi \rangle$ un nombre $r$ de fois, nous obtenons:
\[ Q^r | \psi \rangle = \mathrm{cos}((2r+1)\theta)| \psi_0 \rangle + \mathrm{sin}((2r+1)\theta)| \psi_1 \rangle\]
La probabilité d'obtenir l'état voulu lors de la mesure après $r$ itérations est ainsi $\mathrm{sin}^2 \left( \left( 2r + 1 \right) \theta \right)$ qui est maximisée pour $r \approx \frac{\pi}{4\theta}$. 
\\
Pour $\mathrm{sin}(\theta) \ll 1$, nous pouvons en outre approximer la valeur de $r$ par: \[\frac{\pi}{4\mathrm{sin}(\theta)} = \frac{\pi}{4} \sqrt{\frac{N}{M}} = \mathcal{O}(\sqrt{N})\]
